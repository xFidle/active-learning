\documentclass[12pt, a4paper]{article}
\usepackage[T1]{fontenc}
\usepackage[polish]{babel}
\usepackage[utf8]{inputenc}
\usepackage{indentfirst}
\usepackage{geometry}
\usepackage{amsmath}
\usepackage{hyperref}
\usepackage{algorithm}
\usepackage{algorithmic}
\usepackage{booktabs}
\usepackage{tabularx}
\usepackage[font=footnotesize]{caption}

\geometry{margin=0.75in}
\setlength{\parskip}{6pt}
\linespread{1.25}
\hypersetup{colorlinks=true,linkcolor=blue,urlcolor=blue}

\begin{document}

\title{[UMA] \textendash{} Dokumentacja końcowa projektu \\ \Large Uczenie
	aktywne (temat nr 1) } \author{ Michał Szwejk \and Damian D'Souza } \date{}
\maketitle

\section{Streszczenie założeń dokumnetacji wstępnej} Głównym zadaniem projektu
było przygotowanie programu realizującego schemat uczenia aktywnego. Projekt
zakładał opracowanie dwóch modeli uczenia maszynowego \textendash{} wybrano
\textit{Las Losowy} i \textit{SVM} \textendash{} a następnie porównanie ich
działania w środowisku uczenia aktywnego.

W założeniach wstępnych wyszczególniono także proces przygotowania danych do
treningu. Określono, że testowany zbiór danych zostanie tak spreparowany, aby
stał się on mocno niezrównoważony \textendash{} klasa mniejszościowa powinna
występować kilknaście razy mniej niż większościowa. Pondto wyróżniono, że ze
zbioru danych zostanie usunięte większość etykiet w celu odpowiedniego
przygotowania do procesu uczenia aktywnego.

Jako główną miarę ewaluacji jakości modelu wykorzystano pole pod krzywą
\textit{precision-recall} (PR AUC), ponieważ skupia się ona na jakości predykcji
klasy pozytywnej, a nie na licznych przykładach klasy większościowej.

\section{Uczenie aktywne}
Uczenie aktywne to metoda, która pozwala algorytmom uczenia maszynowego osiągać
lepsze wyniki w przypadkach gdy zbiór danych jest mocno niezbalansowany i
zawiera wiele brakujących etykiet. Model sam, bez ingerencji użytkownika,
wybiera przykłady uczące, które według niego są najbardziej wartościowe dla
treningu. Następnie zewnętrzne źródło (zazwyczaje ekspert domenowy) manualnie
etykietuje wyznaczone próbki. Takie podejście pozwala minimializować potrzebę
posiadania dużego zbioru danych nie poświęcając przy tym skuteczności modelu.

\newpage

Uczenie aktywne jest procesem iteracyjnym. Pseudokod metody może być przedstawiony
następująco:
\begin{algorithm}
	\caption{Uczenie aktywne}
	\begin{algorithmic}[1]
		\STATE{Wybierz początkowy zbiór przykładów trenujących $T$ z całego
			zbioru danych $P$;}
		\STATE{$P := P - T$;}
		\STATE{Poetykietuj przykłady ze zbioru $T$;}
		\REPEAT{}
		\STATE{Przygotuj model $h$ ucząc się na przykładach ze zbioru $T$;}
		\STATE{Wybierz nowe przykłady $Q$ ze zbioru $P$ bazując na wcześniej
			przygotowanym modelu $h$;}
		\STATE{Poetykietuj dane ze zbioru $Q$;}
		\STATE{$T := T \cup Q$};
		\STATE{$P := P \setminus Q$};
		\UNTIL{Poetykietowano 100\%  danych lub spełniono zadane kryterium
			stopu;}
		\RETURN{$h$}.
	\end{algorithmic}
\end{algorithm}

Głownym ograniczeniem uczenia aktywnego jest długotrwały proces etykietowania
danych, który wymaga możliwie jak najepszej ``wyroczni''. Dodatkowo proces może
być kosztowny, ponieważ w każdej iteracji trzeba od nowa dopasowywać model do
danych.


\section{Opis funkcjonalny}
System składa się z dwóch głównych programów uruchamianych z linii poleceń:
\texttt{src.main} realizującego pętlę uczenia aktywnego oraz
\texttt{src.process\_\allowbreak{}images} przygotowującego dane wejściowe na podstawie zdjęć.
Uruchomienie odbywa się przez polecenie
\texttt{python3 -m src.\allowbreak{}\textless{}moduł\textgreater{}}, gdzie nazwa
modułu podawana jest bez rozszerzenia. Przykładowe polecenia:
\begin{itemize}
	\item \texttt{python3 -m src.process\_\allowbreak{}images}
	\item \texttt{python3 -m src.main}
	\item \texttt{python3 -m src.main -c configs/svm\_\allowbreak{}cluster\_\allowbreak{}random}
\end{itemize}

Instalacja zależności może być wykonana w aktywnym środowisku \texttt{venv} przy
pomocy \texttt{pip install}.

Dla \texttt{src.main} można wskazać własny plik konfiguracyjny przy pomocy flagi
\texttt{-c} (lub \texttt{--config}). Domyślnie ładowany jest plik
\texttt{config.toml}, a dla przetwarzania obrazów
\texttt{image\_\allowbreak{}processing\_\allowbreak{}config.toml}.
Jeżeli wskazany plik nie istnieje, system tworzy go automatycznie na podstawie
zarejestrowanych klas konfiguracyjnych.

\subsection*{Konfiguracja przetwarzania obrazów}
Konfiguracja przetwarzania danych steruje pobieraniem zbioru, ekstrakcją cech oraz
przygotowaniem pliku wejściowego do uczenia aktywnego. W pliku znajdują się m.in.:
\begin{itemize}
	\item \texttt{data\_processing.output\_\allowbreak{}dir} \textendash{} katalog zapisu danych
	      (np.\ \texttt{data/active\_\allowbreak{}learning});
	\item \texttt{data\_processing.unlabeled\_\allowbreak{}percentage} \textendash{} odsetek
	      przykładów, których etykiety są ukrywane;
	\item \texttt{data\_processing.majority\_\allowbreak{}ratio} \textendash{} docelowy stosunek
	      klasy większościowej do mniejszościowej;
	\item \texttt{image\_processing.model} \textendash{} nazwa sieci CNN używanej do
	      ekstrakcji cech (np.\ \texttt{resnet50});
	\item \texttt{image\_processing.data\_\allowbreak{}dir} \textendash{} katalog źródłowy ze
	      zdjęciami;
	\item \texttt{image\_processing.force\_\allowbreak{}download} \textendash{} wymuszenie
	      ponownego pobrania danych.
\end{itemize}

Przykładowa konfiguracja:
\begin{verbatim}
[data_processing]
output_dir = "data/active_learning"
unlabeled_percentage = 75
majority_ratio = 20

[image_processing]
model = "resnet50"
data_dir = "data/flowers/images"
force_download = false
\end{verbatim}

\subsection*{Konfiguracja uczenia aktywnego}
Konfiguracja uczenia aktywnego definiuje model klasyfikatora, strategię doboru
próbek oraz parametry walidacji. Dodatkowo dostępny jest moduł
\texttt{python3 -m src.compare\_\allowbreak{with\_}\allowbreak{sklearn}
	\textless{}classifier\textgreater{}}, który uruchamia porównanie z
implementacją \textit{scikit-learn}. Moduł przyjmuje argument
\texttt{classifier} (\texttt{svm} lub \texttt{forest}) oraz korzysta z ustawień
z sekcji \texttt{aggregator} w pliku konfiguracyjnym.
Najważniejsze sekcje:
\begin{itemize}
	\item \texttt{cluster\_initializer} \textendash{} parametry inicjalizacji
	      klastrami (np\. liczba klastrów i proporcje);
	\item \texttt{svm} oraz \texttt{forest} \textendash{} parametry modeli SVM i
	      lasu losowego;
	\item \texttt{active\_learner} \textendash{} wybór klasyfikatora, selektora
	      (\texttt{random}, \texttt{diversity}, \texttt{uncertainty}) i wielkości
	      partii;
	\item \texttt{tester} \textendash{} liczba podziałów, powtórzeń, progi i katalog
	      zapisu wyników;
\end{itemize}

Przykładowa konfiguracja:
\begin{verbatim}
[cluster_initializer]
clusters = 2
center_ratio = 0.6
border_ratio = 0.4

[forest]
n_trees = 100
multiprocessing = false

[forest.tree_config]
max_depth = 10
min_samples_split = 2

[svm]
learning_rate = 0.1
penalty = 100
iter_count = 1000

[active_learner]
classifier = "forest"
selector = "uncertainty"
batch_size = 10
seed = 42
should_store_results = true

[tester]
save_dir = "results"
n_splits = 5
n_repeats = 3
labeled_ratio = 0.25
seed = 42
thresholds = [0.25, 0.3, 0.4, 0.5, 0.75, 1.0]
initializer = "cluster"

[aggregator]
runs = 10
metrics = ["accuracy", "precision", "recall", "f1"]
output = "results/compare"
\end{verbatim}

\section{Opis algorytmów} \subsection{Random forest} Las losowy jest algorytmem,
który generuje wiele różnych drzew w procesie uczenia i przypisuje klasy, będące
dominantą wyników poszczególnych drzew. W odróżnieniu od pojedynczego drzewa
tendencja modelu do nadmiernego dopasowania jest dużo mniejsza. Każde drzewo
trenowane jest na tylu przykładach ile jest w zbiorze trenującym, jednakże
losowane są one ze zwracaniem. Ponadto ograniczony jest zestaw atrybutów,
wybierane jest ich wyłącznie $\sqrt{|A|}$ (bez zwracania), gdzie $A$
\textendash{} to zbiór atrybutów.

Każde drzewo wchodzące w skład lasu losowego jest drzewem typu \textit{CART}.
Reguły podziału w poszczególnych węzłach są konstruowane w taki sposób, aby po
podziale minimalizować wartość \textbf{zanieczyszczenia Gini’ego} (w przypadku
zadania klasyfikacji). Miara ta jest dana wzorem: \begin{equation*} Gini(x) =
	\sum_{i=0}^{|C|} 1-p_i^2 \end{equation*} gdzie $p_i$ \textendash{}
prawdopodobieństwo $i$-tej klasy w węźle, a $C$ \textendash{} zbiór klas.

\subsection{SVM} Maszyna wektorów nośnych jest algorytmem klasyfikacji, który
wyznacza optymalną hiperpłaszczyznę rozdzielającą klasy w przestrzeni cech. Dla
problemu binarnego hiperpłaszczyzna jest definiowana przez wektor wag
$\mathbf{w}$ i wyraz wolny $b$, a klasyfikacja nowej próbki $\mathbf{x}$ odbywa
się na podstawie znaku funkcji decyzyjnej $f(\mathbf{x}) =
	\mathbf{w}^T\mathbf{x} + b$.

Kluczową ideą SVM jest maksymalizacja marginesu, czyli odległości między
hiperpłaszczyzną a najbliższymi punktami każdej z klas (tzw.\ wektorami
nośnymi). W przypadku danych nieliniowo rozdzielnych wprowadza się parametr kary
$C$, który kontroluje kompromis między maksymalizacją marginesu a minimalizacją
błędu klasyfikacji. W implementacji zastosowano standaryzację cech (średnia 0,
odchylenie 1), co stabilizuje uczenie i pozwala zachować porównywalną skalę
atrybutów. Uczenie wag realizowane jest metodą SGD w wariancie Pegasos z
malejącym krokiem uczenia oraz tasowaniem próbek w każdej epoce, minimalizując
funkcję celu zawierającą \textbf{funkcję straty zawiasowej}:

\begin{equation*} L(\mathbf{w}, b) = \frac{1}{2}\|\mathbf{w}\|^2 + C
	\sum_{i=1}^{n} \max(0, 1 - y_i(\mathbf{w}^T\mathbf{x}_i + b)) \end{equation*}

\noindent gdzie $y_i \in \{-1, 1\}$ \textendash{} etykieta $i$-tej próbki, a $C$
\textendash{} parametr kary.

W sytuacjach wymagających probabilistycznego wyjścia modelu (co jest kluczowe
np.\ w uczeniu aktywnym), możliwe jest zastosowanie metody \textbf{Platta}.
Pozwala ona na dopasowanie funkcji sigmoidalnej do wartości funkcji decyzyjnej
SVM, co umożliwia przekształcenie ich w oszacowania prawdopodobieństwa:

\begin{equation*} P(y=1|\mathbf{x}) = \frac{1}{1 + \exp(A \cdot f(\mathbf{x}) +
		B)} \end{equation*}

\noindent gdzie parametry $A$ i $B$ wyznaczane są poprzez minimalizację ujemnej
logarytmicznej funkcji wiarygodności na zbiorze treningowym.

\section{Zbiory danych}
\subsection{Pomocniczy} W ramach prostych testów sprawdzających poprawność
implementacji naszych algorytmów wykorzystano odniesienie do wartości zwracanej
przez klasyfikatory opracowane w bilbliotece \textit{SKLEARN}. Do testów
wykorzystano zbiór danych dotyczący
\href{https://archive.ics.uci.edu/dataset/109/wine}{jakości win}. Zadanie
sprowadzono do klasyfikacji binarnej \textendash{} jakość mniejsza niż 6 wskazuje na
klasę 0, a większa na 1. Zbiór zawiera 1599 przykładów i 11 atrybutów, z czego
wszystkie są ciągłe.

\subsection{Główny}
Jako główny zbiór danych wykorzystywany w pętli uczenia aktywnego wybrano
\href{https://www.kaggle.com/datasets/imsparsh/flowers-dataset/data}{zestaw}
zawierający zdjęcia kwiatów. Pierwotnie podjęto taką decyzję, ponieważ założono,
że przykłady będą etykietowane manualnie co jest łatwiejsze do zrealizowania
patrząc na zdjęcie, a nie wektor cech. Jak się później okazało, takie podejście
jest niemożliwe do zrealizowania ze względów ograniczeń czasowych jeżeli chcemy
przeprowadzić dokładne eksperymenty bazujące na $n$-krotnych powtórzeniach
walidacji krzyżowej. Przed przystąpieniem do przeprowadzenia eksperymentów
został zbiór został przetworzony w następujący sposób:

\begin{itemize}
	\item Pozostawienie wyłącznie próbek zklasyfikowanych jako mniszki lekarskie
	      lub słoneczniki \textendash{} ograniczenie się do klasyfikacji binarnej;
	\item Usunięcie prawie wszystkich próbek słoneczników (ostateczny stosunek
	      klas to 1:20) \textendash{} stworzenie niezbalansowanego zbioru danych;
	\item Zamaskowanie etykiet 75\% przykładów (stają się one całkowicie
	      niewidoczne dla modelu) \textendash{} symulacja sytuacji braku etykiet;
	\item  Przetworzenie zdjęcia do postaci reprezentacji wektorowej z
	      wykorzystaniem wcześniej wytrenowanej sieci neuronowej \textit{ResNet50}.
\end{itemize}

Ostatecznie zbiór danych składa się z 680 przykładów \textendash{} 30
słoneczników (klasa 1) i 650 mniszków lekarskich (klasa 0). Każdy przykład ma
dokładnie 2048 atrybutów (wszystkie ciągłe).

\section{Eksperymenty}
\subsection{Porównanie z \textit{SKLEARN}} Tak jak wspomniano w sekcji
dotyczącej opisu zbiorów danych przeprowadzono testy porównawcze naszych
algorytmów z bilbioteką \textit{SKLEARN} w celu odniesienia uzyskanych wyników
do wartości referencyjnych. Przeprowadzonych zostało 10 niezależnych
uruchomień, każde z innym ziarnem startowym wykorzystywanym do dzielenia
zbiorów danych na trenujący i testujący (80\% \textendash{} 20\%) oraz jako
parametr algorytmu uczącego. Następnie zagregowano wyniki. Parametry obu modeli
zostały ustawione jednakowo, zgodnie z domyślnymi w \textit{SKLEARN}. Do
porównania wykorzystano standardowe miary oceny jakości klasyfikatora.

\begin{table}[!ht]
	\begin{minipage}{0.48\linewidth}
		\centering
		\small
		\begin{tabular}{lcccc}
			\toprule
			algorytm                  & średnia & std   & max   & min   \\
			\midrule
			forest                    & 0.777   & 0.009 & 0.790 & 0.761 \\
			\textit{\textbf{SKLEARN}} & 0.800   & 0.004 & 0.807 & 0.790 \\
			\bottomrule
		\end{tabular}
		\caption{Porównanie miary \textbf{accuracy}.}
	\end{minipage}
	\hfill
	\begin{minipage}{0.48\linewidth}
		\centering
		\small
		\begin{tabular}{lcccc}
			\toprule
			algorytm                  & średnia & std   & max   & min   \\
			\midrule
			forest                    & 0.8404  & 0.006 & 0.848 & 0.828 \\
			\textbf{\textit{SKLEARN}} & 0.8472  & 0.003 & 0.852 & 0.837 \\
			\bottomrule
		\end{tabular}
		\caption{Porównanie miary \textbf{F1-Score}.}
	\end{minipage}
\end{table}

\begin{table}[!ht]
	\begin{minipage}{0.48\linewidth}
		\centering
		\small
		\begin{tabular}{lcccc}
			\toprule
			algorytm         & średnia & std   & max   & min   \\
			\midrule
			\textbf{forest}  & 0.925   & 0.009 & 0.935 & 0.905 \\
			\textit{SKLEARN} & 0.874   & 0.009 & 0.884 & 0.855 \\
			\bottomrule
		\end{tabular}
		\caption{Porównanie miary \textbf{recall}.}
	\end{minipage}
	\hfill
	\begin{minipage}{0.48\linewidth}
		\centering
		\small
		\begin{tabular}{lcccc}
			\toprule
			algorytm                  & średnia & std   & max   & min   \\
			\midrule
			forest                    & 0.7697  & 0.007 & 0.783 & 0.757 \\
			\textbf{\textit{SKLEARN}} & 0.8219  & 0.005 & 0.834 & 0.814 \\
			\bottomrule
		\end{tabular}
		\caption{Porównanie miary \textbf{precision}.}
	\end{minipage}
\end{table}

Wyniki są porównywalne. Różnice mogą wynikać z różnic w poszczególnych
implementacjach algorytmów. Jednakże wyniki są na tyle blisko, że bez straty
ogólności można założyć, że algorytmy zostały zaimplementowane poprawnie.

\subsection{Proces uczenia aktywnego}

\section{Opis wykorzystanych narzędzi}
W implementacji wykorzystano narzędzia i biblioteki wspierające zarówno obróbkę
obrazów, jak i eksperymenty z uczeniem aktywnym. Projekt został napisany w języku
Python 3.13, a środowisko uruchomieniowe może być skonfigurowane w ramach
\texttt{venv} z instalacją zależności przez \texttt{pip}.

Najważniejsze użyte technologie i biblioteki:
\begin{itemize}
	\item \texttt{numpy} i \texttt{pandas} \textendash{} przetwarzanie danych i
	      przygotowanie macierzy cech;
	\item \texttt{scikit-learn} \textendash{} metryki jakości, podziały danych oraz
	      implementacje referencyjne do porównań;
	\item \texttt{torch} i \texttt{torchvision} \textendash{} ekstrakcja cech z obrazów
	      przy użyciu sieci \textit{ResNet50};
	\item \texttt{kagglehub} \textendash{} pobieranie zbioru zdjęć kwiatów;
	\item \texttt{ucimlrepo} \textendash{} pobieranie zbioru jakości win do testów
	      porównawczych;
	\item \texttt{matplotlib} \textendash{} wsparcie dla wizualizacji wyników;
\end{itemize}

\end{document}

